%% This is file `example_DarkConsole.tex',
%% generated with the docstrip utility.
%%
%% The original source files were:
%%
%% examples_kmbeamer.dtx  (with options: `DarkConsole')
%% Copyright (c) 2011-2013 Kazuki Maeda <kmaeda@users.sourceforge.jp>
%% 
%% Distributable under the MIT License:
%% http://www.opensource.org/licenses/mit-license.php
%% 

%%% もし pdfTeX や LuaTeX を使うなら dvipdfmx オプションを外す.
% \documentclass[dvipdfmx]{beamer}

% Modified by LianTze Lim to work with fontspec/xelatex
\documentclass{beamer}
% \usepackage{mathspec}
% \usepackage{xeCJK}
% \setCJKmainfont{IPAPMincho}
% \setCJKsansfont{IPAGothic}
% \setCJKmonofont{IPAGothic}

% You can set fonts for Latin script here
% \setmainfont{FreeSerif}
% \setsansfont{FreeSans}
% \setmonofont{Latin Modern Mono}

\usetheme{DarkConsole}

% Saulo's packages
\usepackage{graphicx}
\usepackage{media9}
\usepackage{listings,lstautogobble}
\usepackage{xcolor}
\usepackage{rotating}
\usepackage{MnSymbol,wasysym}
\usepackage{pgffor}
\usepackage{ragged2e}   %for justification
% \usepackage{xmpmulti}
% \usepackage{animate}[2017/05/18]
\usepackage{eso-pic}
\beamertemplatenavigationsymbolsempty

\newcommand\AtPagemyUpperLeft[1]{\AtPageLowerLeft{%
\put(\LenToUnit{0.95\paperwidth},\LenToUnit{0.93\paperheight}){#1}}}
\AddToShipoutPictureFG{
  \AtPagemyUpperLeft{{\includegraphics[width=.5cm,keepaspectratio]{../logo2.jpg}}}
}%

\newcommand{\comma}{,}

\newcommand{\myduration}[1]{%
    \foreach \l in {#1}{%
        \transduration<\l>{.025}%
    }%
}

\newcommand{\anitext}[2]{%
    \myduration{#1}%
    \foreach \i in {#2}{%
        \temporal<+>{}{\Large\bfseries \vphantom{Mpgjy}\i}{\vphantom{Mpgjy}\i}%        %the phantom removes the bumping
    }%
}


\lstset{ %
    backgroundcolor=\color{black},   % choose the background color
    basicstyle=\footnotesize,        % size of fonts used for the code
    numberstyle=\color{green},
    breaklines=true,                 % automatic line breaking only at whitespace
    captionpos=t,                    % sets the caption-position to bottom
    commentstyle=\color{gray},    % comment style
    escapeinside={\%*}{*)},          % if you want to add LaTeX within your code
    stringstyle=\color{red},     % string literal style
    otherkeywords={keras}, 
    morekeywords=[2]{compile},% Add keywords here
    keywordstyle=\color{yellow},
    keywordstyle=[2]{\color{cyan}},
    emph={Sequential, layers, Dense, fit, predict},          % Custom highlighting
    emphstyle=\color{cyan},    % Custom highlighting style
    showstringspaces=false,
    autogobble=true,
    literate=%
    {0}{{{\color{green}0}}}1
    {1}{{{\color{green}1}}}1
    {2}{{{\color{green}2}}}1
    {3}{{{\color{green}3}}}1
    {4}{{{\color{green}4}}}1
    {5}{{{\color{green}5}}}1
    {6}{{{\color{green}6}}}1
    {7}{{{\color{green}7}}}1
    {8}{{{\color{green}8}}}1
    {9}{{{\color{green}9}}}1
}

%%% もし pTeX + dvipdfmx を使うならば以下のどちらかを環境に合わせてコメントアウト.
%% \AtBeginDvi{\special{pdf:tounicode EUC-UCS2}} % EUC の場合
%% \AtBeginDvi{\special{pdf:tounicode 90ms-RKSJ-UCS2}} % SJIS の場合

%%% もし LuaTeX で日本語を出力するなら以下をコメントアウト.
%% \usefonttheme{luatexja}
%% \hypersetup{unicode}

%%% 日本語を使うなら以下を入れると定理環境中のフォントが立体になる.
%%% 欧文なら不要.
%%% LLT: Comment out this line if your presentation is in English or other European languages
\setbeamertemplate{theorems}[normal font]

\title{\texttt{Introduction to \textbf{data visualization} with Python}}
\subtitle{Women in Python}
\author{Saulo Meirelles} %\footnote{\texttt{WeChat ID: Saulomini}}}

\begin{document}

\begin{frame}
  \maketitle
\end{frame}

\begin{frame}{Learning objetives}

    \begin{enumerate}
    \item Understand the importance of data visualization.\pause
    \item Create figures (graphs) in Python.\pause
    \item Basic data treatment.\pause
    \end{enumerate}

\end{frame}

\begin{frame}{Definition}

\centering  
  Data visualization is the graphical representation of information and data. 

  \begin{figure}
  \includegraphics[width=.5\textwidth]{../datavis.jpeg}
  \end{figure}
  
\end{frame}


\begin{frame}{The importance of data visualization}

  \begin{figure}
  \includegraphics[width=.7\textwidth]{../tablevscharts.jpg}
  \end{figure}
  \begin{figure}
  \includegraphics[width=.7\textwidth]{../datavis2.jpeg}
  \end{figure}

\end{frame}


\begin{frame}{The importance of data visualization}
 
\begin{columns}

\begin{column}[t]{0.55\textwidth}


  \begin{figure}
  \includegraphics[width=\textwidth]{../tablevscharts.jpg}
  \end{figure}
  \begin{figure}
  \includegraphics[width=\textwidth]{../datavis2.jpeg}
  \end{figure}

\end{column}

\begin{column}[t]{0.4\textwidth}

    \begin{itemize}[<+(1)->]

    \item Information is quickly absorved.
    \item Find trends and outliers.
    \item Identify relationships and patterns.
    \item Communicate the story to others.
    \item Handle big data.


    \end{itemize}

\end{column}

\end{columns}
  
\end{frame}


\begin{frame}{Types of charts}

  \begin{figure}
  \includegraphics[width=\textwidth]{../typesofcharts.png}
  \end{figure}
  
\end{frame}


\begin{frame}[fragile]{Hands on!}

        \begin{lstlisting}[language= Python, frame = single, title={Packages to install}]
        conda install -c conda-forge matplotlib 
        \end{lstlisting}
        
\end{frame}


\end{document}

\endinput
%%
%% End of file `example_DarkConsole.tex'.
